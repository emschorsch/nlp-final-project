\documentclass[11pt]{article}
\usepackage{cs65f12}
\usepackage{times}
\usepackage{latexsym}
\usepackage{amsmath}
\usepackage{amssymb}
\usepackage{float}
\usepackage{pgfplots}
\newcommand{\cou}{\textrm{count}}
\setlength\titlebox{6.5cm}    % Expanding the titlebox

\title{Sentiment Classification and Analysis in Twitter}

\author{Justin Cosentino\\
  Department of Computer Science\\
  Swarthmore College\\
  Swarthmore, PA 19081\\
  {\tt jcosent1@swarthmore.edu}  
  \And                            
  Emanuel Schorsch\\                 
  Department of Computer Science\\
  Swarthmore College\\
  Swarthmore, PA 19081\\
  {\tt eschors1@swarthmore.edu}}

\date{}

\begin{document}
\maketitle
\begin{abstract}
As the popularity of microblogging drastically increases, it is evident that there is a vast amount of opinionated data and information available to assess the general sentiment of millions of Twitter users in regards to products, events, and people. We consider the problem of classifying as either positive, negative, neutral, or objective, the overall sentiment of a tweet and the sentiment of a marked instance of a word or phrase within a given tweet. 

Using two Twitter corpora provided by {\tt SemEval-2013}, we implemented and trained a naive Bayes classifier, a decision list classifier, and a subjectivity lexicon classifier. Our data suggests that the accuracy of our models will further increase as the size of our training corpora increases. We also find that the naive Bayes and decision list classifiers, when trained on a sufficiently large corpus, perform far superior to the subjectivity lexicon classifier. We find that the presence of slang and abbreviations complicates efforts to incorporate previously built sentiment lexicons. Finally, we discuss the issues surrounding the heavy skew of the polarity of tweets towards the most frequent sense.
\end{abstract}

\section{Introduction}
To answer the question of how a given population feels about an issue it is usually necessary to conduct some kind of poll. This is neither cheap nor quick. Ideally, we would instead be able to harvest the abundant information on the internet to find our answer. 

In recent years we have seen an explosion in social media use. Facebook and Twitter are two prime examples of platforms which millions of different people use on a daily basis. These platforms are perfectly positioned to answer the kinds of questions previously addressed by polls. Since everyone from your neighbor to Obama is using these social networking platforms we have access to the opinions of a diverse sample of individuals. We choose to focus on Twitter as it is geared exclusively towards the expression of people's thoughts.  

Twitter is a microblogging platform which allows users to broadcast messages (tweets) of up to 140 characters to their followers. Since tweets must be so short they are almost always about a single topic, with usually only a single sentiment expressed. While Twitter does have a significant portion of non-English speaking users, for the purposes of this paper we restrict our attention to tweets in english. A useful feature of Twitter is that tweets are labeled by hashtags, which are keywords preceded by \#'s. This makes it simple to search for a specific hashtag such as \#Toyota, which will yield all tweets that are self-classified as relating to Toyotas. Given the enormous volume of tweets per day (340 million\footnote{{\tt http://blog.twitter.com/2012/03/twitter{\\}-turns-six.html}}) it is possible to find a sufficiently large sample of tweets about most topics, even relatively obscure ones.

There remain issues with ensuring that Twitter users represent a sufficiently random population but the issue of automated sentiment extraction looms much larger. In this paper we first discuss how to capture the sentiment of a subset of a tweet, either a word or a phrase. We refer to this problem as taskA. This is useful as a building block which can hopefully aid in deciphering the sentiment of a tweet as a whole. Naturally, we then consider the tweet as a whole and the overall sentiment of the tweet with respect to a given topic. We refer to this second problem in the paper as taskB. A system which could perform well on this task would be invaluable for anyone studying public opinion. This would be immediately applicable to product managers and campaign managers as well as many researchers.

We found that simple features were sufficient to perform significantly above the baseline. It was interesting to see that having statistics on the prevalence of individual words among different senses was very powerful. We will discuss in the paper three different linguistic techniques we used to categorize tweets into their appropriate sense.

\section{Related Works}
Sentiment analysis is a rapidly expanding field of natural language processing and a vast majority of these studies attempt to tackle a combination of two issues found in sentiment analysis: the identification of sentiment and then the classification of the identified sentiment.  Many experiments attempt to classify the polarity of larger documents, specifically online reviews. For example, Pang and Lee analyze the sentiment of online movie reviews~\cite{pang2002thumbs}. Using machine learning techniques, they study the effectiveness of a variety of features on sentiment detection. Their studies show that the presence of n-gram models is most useful in determining the polarity of a given review. Additionally, Pang and Lee found that incorporating parts of speech tagging and the presence of adjectives had little  if not a negative effect on the accuracy of their models. Accordingly, we have chosen not to experiment with features involving parts of speech. Pang and Lee also chose to experiment with word position. However, because the average tweet contains only eleven words, this is thought to have little effect on determining sentiment and will thus also be ignored~\cite{oconnor2010tweets}.

There have also been many recent studies that have attempted to analyze the sentiment contained within tweets and other microblogging message formats. There have been various applications of Twitter sentiment classification, ranging from an analysis of the usage of microblogging as a tool for consumer opinion sharing, to showing the relation between Twitter sentiment and political public opinion~\cite{jansen2009twitter,oconnor2010tweets}. Microblogging features, such as hash-tags and emoticons, have played a key role in this analysis. Davidov et al.~\shortcite{davidov2010enhanced} proposed a supervised sentiment classification framework that utilized these features. Using hand tagged hash-tags and emoticons paired with feature types such as punctuation, patterns, and n-grams, they showed that by using a k-nearest neighbors strategy tweets can be paired with one or more of their labeled hash-tags, returning the overall sentiment of that tweet. 
\section{Methodology}

\subsection{Data}
A total of three different corpora were used in our experiments. Two of these corpora contained Twitter data that was used in the training and testing of our classifiers. The third data set was a sentiment lexicon which was used to tag the polarity of words.

\subsubsection*{Twitter Corpus}
As previously stated, Twitter is a microblogging service that allows users to post short, 140-character messages called tweets.  The two Twitter corpora used in this study were acquired from the {\tt SemEval-2013} Competition site\footnote{Available at {\tt http://www.cs.york.ac.uk/{\\}semeval-2013/task2/}} using the provided script, which downloaded the tweets from the Twitter webpage. Some of these tweets were no longer available due to a user changing their privacy settings, deleting the designated tweet, or deleting their account. Each corpus contained tweets covering a wide range of topics including entities, products, and events. The tweets found in the corpora were also exclusively written in English. The first corpus contained full tweets and the polarity tag for a marked instance of a word or phrase within each tweet. The second corpus contained full tweets and the polarity tag of a given topic found within the content of each tweet. There were a total of 2,377 tweet segments in the first corpus and 592 full tweets in the second. The polarity tags for each corpus were limited to 'positve', 'negative', 'neutral' and 'objective'. Although some of the tweets contained within each corpus were identical, the first corpus prompted users to only look at the given segment of the tweet. Each data set was used for both training and testing using cross-validation.      

\subsubsection*{Subjectivity Lexicon}
The third corpus used in our experiments contained subjectivity data on roughly 6,500 words. This lexicon was acquired from OpinionFinder, a system that performs subjectivity analysis~\cite{wilson2005opinionfinder}. Each word within this corpus has a corresponding polarity, strength of subjectivity, part-of-speech tag, stem value, and word length. We ignored the strength of subjectivity, stem value, part-of-speech-tag, and word length. The polarity tag was used as a means of classifying the sentiment content of tweets.

\subsection{Features}
In order to use the sentiment classifiers to label the given data, we needed to feed in features, which we decided to build using the bag-of-words model. This model is implemented as an unordered listing of features and does not take into account word order or word location. Using this model we create feature vectors that contain n-gram features present in a given tweet. The frequency of these features is also represented by the vector. In the implementation of the decision list and naive Bayes classifiers, unigrams, bigrams, and trigrams were used to train and classify data. In the subjectivity lexicon classifier, the words in the tweet and their associated polarity made up the features in the vector. 

\subsection{Sentiment Classifiers}
In order to determine the sentiment of a given tweet, three algorithms were implemented: a naive Bayes classifier, a decision list classifier, and a subjectivity lexicon classifier.
\subsubsection*{Naive Bayes}
Using the Naive Bayes classifier, a given feature vector $\vec{f}$ is assigned
the polarity \^{s} given by: 
$$\hat{s} = \underset{{s}\in{S}}{\arg\max}P(s|\vec{f})$$
Using Bayes rule we can rewrite this equation:
\[\hat{s} = \underset{{s}\in{S}}{\arg\max}{\frac{P(\vec{f}|s)P(s)}{P(\vec{f})}}.\]
Since $P(\vec{f})$ is constant across all polarities, it
does not affect the choice of $\hat{s}$ and we can discard it. This leaves us with:
\[\hat{s} = \underset{{s}\in{S}}{\arg\max}{P(\vec{f}|s)P(s)}.\]

In order to make the calculation of $P(\vec{f}|s)$ comutationally simple we naively assume that each of the features in the vector are conditionally independent of each other. Under this assumption the below equality holds:
\[P(\vec{f}|s) = \prod_{i=1}^n P(f_i|s).\]

The probability of the vector $\vec{f}$ under the assumption of independence is equal to the product of the independent probability of each feature in the vector. Substituting into our previous equation we arrive at our naive Bayes classifier:
\[\hat{s} = \underset{s \in S}{\arg\max}{{\prod_{i=1}^n}P(f_i|s)}P(s).\]

The probability of ${f_i}$ which we use is the maximum likelihood estimate. This is computed by the number of times $f_i$ occurs as sense $s$ over the number of times $f_i$ occurs total. 
This is represented by:
\[P(f_i|s_i) = \frac{\cou(f_i,s)}{\cou(f_i)}.\]

The probability of a specific sense $s$ is calculated in a similar manner:
$$P(s) = \frac{\cou(s)}{\underset{s_i \in S}{\sum} \cou(s_i)}.$$

However, in our model we do not multiply by $P(s)$. In our model we use the modified formula:
\[\hat{s} = \underset{s \in S}{\arg\max}{{\prod_{i=1}^n}P(f_i|s)}.\]

The feature we relied on most heavily was a unigram model. However, we also implemented bigram and trigram models to see if adding those features to the naive Bayes classifier would improve performance. While higher order n-gram models blatantly violate the assumption of conditional indpendence the naive Bayes classifier still performs well.

\subsubsection*{Decision List}
A decision list classifier is equivalent to an extended if-else statement. Decision lists generate a set of rules or conditions, for which there is a single classification associated. These rules are generated from tagged feature vectors and then scored and ordered based on their associated scores. Similar to the approach used by Yarowsky, each feature and value pair is treated as a rule~\cite{yarowsky1994decision}. In order to generate the best rule for each possible classification, the following equation is used:
\[ \log\left({\frac{P(S_i|f_i)}{P(\text{All other senses}|f_i)}}\right) \]
\indent One important difference to note about our system is that our smoothing method is similar to but not identical to add-k smoothing. The smoothing method which we used was to add a constant, $\alpha = 1$, to the denominator and numerator of the scoring equation. This stands in contrast to add-k smoothing which is similar but $\alpha V$ is added to the denominator. Once these rules have been generated from the given feature sets, the decision list creates a heirarchy of rules similar to Table 1, with the highest scoring rules at the top. The decision list will start at the top and use the highest rule which is satisfied by the given feature. If the next rule's score is below one the decision list will default to classifying with the most frequent sense, as extracted from a training corpus.

\begin{table}[H]
  \begin{center}
  \begin{tabular}{| p{3cm} l l |}
  \hline
  Rule & & Polarity \\ \hline
  love & $\Rightarrow$ & Positive \\
  can't wait & $\Rightarrow$ & Positive \\
  looking forward & $\Rightarrow$ & Positive \\
  confirmed & $\Rightarrow$ & Objective \\
  crash & $\Rightarrow$ & Negative \\
  ... & $\Rightarrow$ & ... \\
  Score $<$ 1 & $\Rightarrow$ & MFS \\ \hline
  \end{tabular}
  \end{center}
  \caption{Example rules generated by the decision list after training on the second Twitter corpus.}
\end{table}

\subsubsection*{Subjectivity Lexicon}
A subjectivity lexicon was also used to determine the polarity of tweets. The subjectivity lexicon, a list containing roughly 6,500 words and their polarity, was from OpinionFinder. This lexicon was used to determine the polarity of each word in the given tweet or tweet segment.

\indent Two variations of the subjectivity classifier were implemented. In both variations, individual counts of the number of positive and negative words for each tweet were kept. If a tweet or tweet segment contained more positive words than negative words, the tweet was classified as positive. If the tweet contained more negative words than positive words, the tweet was classified as negative. In the first classifier, if a tweet had no positive or negative words or if the count of positive and negative words were equal, the tweet was labeled as objective. The first subjectivity lexicon classifier never used the neutral label. In the above case the second subjectivity lexicon classifier would classify the tweet with the most frequent sense, as extracted from the training data. This second classifier will be referred to as the MFS subjectivity lexicon classifier.

\indent Because the first subjectivity classifier requires no labeled tweets to be used as training data, this classifier was tested on all tweets in each corpus. However, in order to compare the results of the subjectivity lexicon classifier to the results of our other classifiers, the lexicon was also used to label only the test data used by all other classifiers. The MFS subjectivity lexicon classifier was trained and tested on the same data as the decision list and naive Bayes classifiers. 

\subsubsection*{Most Frequent Sense}
The most frequent sense (MFS) classifier is used as a baseline for comparison in our experiments. For each corpus, the classifier counts the occurrences of each sense. The classifier then labels all test data as the sense with the highest count.

\section{Results}
\begin{table*}[htb!]
  \centering
  \begin{tabular}{| l || l | l | l | l | l |}
  \hline
  Features & Naive Bayes & Decision List & Subj. Lexicon & MFS Subj. Lexicon & MFS \\ 
  \hline \hline
  Unigrams              & \bf{59.32}\%      & 58.66\% & 52.28\% & 55.47\% 
                        & 52.12\% \\  \hline
  Bigrams               & \bf{58.13}\%      & 52.12\%      & N/A          
                        & N/A          & 52.12\% \\  \hline
  Trigrams              & \bf{52.20}\%      & 52.12\%      & N/A          
                        & N/A          & 52.12\% \\  \hline
  Bigrams \& Unigrams   & \bf{59.66}\% & 58.49\%      & N/A          
           & N/A          & 52.12\% \\  \hline
  Bi, Tri, \& Unigrams  & 58.48\%      & \bf{58.49}\%      & N/A          
           & N/A          & 52.12\% \\  \hline
  \end{tabular}
  \caption{Average accuracy of classifiers on taskB using fivefold cross-validation on the second Twitter corpus.}
\end{table*}

\begin{table*}[htbp!]
  \centering
  \begin{tabular}{| l || l | l | l | l| l |}
  \hline
  Features & Naive Bayes & Decision List & Subj. Lexicon & MFS Subj. Lexicon & MFS \\ 
  \hline \hline
  Unigrams              & \bf{64.84}\%      & 64.12\% & 44.98\% & 44.98\%
                        & 64.12\% \\  \hline
  Bigrams               & \bf{64.76}\%      & 64.12\%      & N/A      
                        & N/A          & 64.12\% \\  \hline
  Trigrams              & \bf{64.38}\%      & 64.12\%      & N/A          
                        & N/A          & 64.12\% \\  \hline
  Bigrams \& Unigrams   & \bf{65.56\%} & 64.12\%      & N/A          & N/A          & 64.12\% \\  \hline
  Bi, Tri, \& Unigrams  & \bf{65.56\%} & 64.12\%      & N/A          & N/A          & 64.12\% \\  \hline  
  \end{tabular}
  \caption{Average accuracy of classifiers on taskA using fivefold cross-validation on the first Twitter corpus.}
\end{table*}

\subsection{Tweet Polarity Classification}
The classification accuracies for determining the polarity of a tweet with respect to a given topic are shown in Table 2. These entries represent the average accuracy of each classifier when trained with the given features using fivefold cross-validation. It is evident that bag-of-unigram features provided the best classification results, with each classifier surpassing the most frequent sense baseline of 52.12\%.

\subsubsection*{Decision List}
The decision list classifier did very well when training on unigrams. However, when trained solely on bigrams or solely on trigrams it performed identically to the most frequent sense baseline. When we combined unigram features with bigram or trigram features, the accuracy of the decision list performed on par with a decision list trained on unigram features. Varying the alpha, the value used in smoothing, had no significant effect on classification accuracy. 

\subsubsection*{Naive Bayes}
The naive Bayes classifier was the most consistent classifier and generated the highest average accuracies. When trained on unigram feature sets, this classifier had similar accuracies to the decision list classifier. However, this classifier was able to score much better than both the decision list and the most frequent sense baseline when trained on bag-of-bigram features.

\subsubsection*{Subjectivity Lexicon}
The subjectivity lexicon had the lowest accuracy of the three classifiers, scoring just above the most frequent baseline. Although the lexicon was able to somewhat correctly determine the polarity of a given word, it was not able to account for sarcasm or negation. This resulted in the incorrect classification of tweets. It is worth noting that the most frequent sense subjectivity classifier had a better average accuracy than both the most frequent sense and subjectivity lexicon classifiers, but not as high as either the naive Bayes nor decision list classifiers. 

\subsection{Contextual Polarity Classification}
The classification accuracies for classifying a given tweet segment are shown in Table 3. These results were significantly worse than the accuracy of these classifiers in taskB. Both the decision list and naive Bayes classifiers had an accuracy approximately equal to the most frequent sense baseline. The accuracies of the subjectivity lexicon and MFS subjectivity lexicon classifiers were far below this baseline. Varying the features each model trains on, whether unigrams, bigrams, trigrams or some combination, has only a small effect on the accuracy. Each entry within Table 3 represents the average accuracy from fivefold cross-validation when each classifier was trained on the specified features. 

\subsubsection*{Decision List}
The decision list classifier was equal with the most frequent sense classifier when it came to determining the polarity of a given subsection of a tweet. Differing the features the classifier was trained on did not result in any differences in this accuracy. Additionally varying alpha, the value used for smoothing, had little effect.  

\subsubsection*{Naive Bayes}
The naive Bayes classifier was the only classifier able to score above the most frequent sense baseline. However, the accuracy of this classifier was barely above the baseline accuracy of 64.12\%. Thus, although this classifier scored slightly higher than the decision list, it is no way an effective means of classifying the polarity of a given segment of a tweet.

\subsubsection*{Subjectivity Lexicon}
Both the subjectivity lexicon and MFS subjectivity lexicon scored far below the baseline of the most frequent sense classifier. Since the most frequent sense for this data set was objective, both classifiers, had they found no sentiment in the tweet or found an equal number of positive and negative words, would have labeled the polarity of the tweet segment as objective. 

\begin{figure}[htb!]
  \centering
  \begin{tikzpicture}
  \begin{axis}[
    xlabel=Training Size (Percent of Corpus),
    ylabel=Accuracy,
    width = 7cm,
    height = 10cm
  ]
  \addlegendentry{Most Freq. Sense};  
  \addplot[color=red, mark=*] coordinates {
    (10,46.67)
    (20,46.67)
    (30,46.67)
    (40,46.67)
    (50,46.67)
    (60,46.67)
    (70,46.67)
    (80,46.67)
  };
  \addlegendentry{Subj. Lexicon};  
  \addplot[color=blue, mark=x] coordinates {
    (10,54.17)
    (20,54.17)
    (30,54.17)
    (40,54.17)
    (50,54.17)
    (60,54.17)
    (70,54.17)
    (80,54.17)
  };
  \addlegendentry{MFS Subj. Lexicon};  
  \addplot[color=yellow, mark=*] coordinates {
    (10,53.33)
    (20,53.33)
    (30,53.33)
    (40,53.33)
    (50,53.33)
    (60,53.33)
    (70,53.33)
    (80,53.33)
  };
  \addlegendentry{DL: Unigrams};  
  \addplot[color=green, mark=+] coordinates {
    (10,45.83)
    (20,45.83)
    (30,52.50)
    (40,49.17)
    (50,52.50)
    (60,55.83)
    (70,57.50)
    (80,58.33)
  };
  \addlegendentry{NB: Unigrams};  
  \addplot[color=black, mark = diamond] coordinates {
    (10,43.70)
    (20,50.42)
    (30,46.22)
    (40,46.22)
    (50,52.10)
    (60,52.94)
    (70,54.62)
    (80,57.14)
  };
  \addplot[color=white] coordinates {
    (10,65)
  };
  \end{axis}
  \end{tikzpicture}
  \caption{Accuracy of classifiers labeling the sentiment of a given topic contained within a tweet. Each classifier was trained on the indicated percent of the second Twitter corpus and were then tested on the last 20\% of the data.}
\end{figure}

\subsection{Effects of Limited Training on Accuracy}
In addition to analyzing the ability of each classifier to determine the polarity of subjects and segments of tweets, we experimented with limiting the amount of training data available to each classifier. As shown in Figure 1, each classifier was given a percentage of training data from the second corpus and then used to classify the polarity of a designated test data set. For this experiment, twenty-percent of the tweets from the second Twitter corpus were reserved as test data. Each classifier was trained on the first $x$ percentage of the corpus, starting at $10$ and increasing by ten-percent each run. The most frequent sense and both subjectivity lexicon classifiers remained at a constant accuracy as they were not trained. We can see an increase in accuracy in both the naive Bayes and decision list classifiers as the effective size of the training data increases. This suggests that if we train each model on more data, which we do not currently have, their accuracy will continue to rise. 

\section{Analysis}
\indent The results we produced using our machine learning algorithms were superior to the most frequent sense baseline in taskB. However, in taskA the vast majority of our systems only equaled the most frequent sense. One reason for this is that the subset of tweets which we are classifying in taskA are simply too small. This is especially a factor in the decision list. The decision list will be looking for rules with a sufficiently high score. Since there are so few features in each subset it is highly probably that no rule will be triggered. This becomes even more likely when considering the size of our data and hence the smaller number of high quality rules. The decision list defaults to the most frequent sense, and therefore, since the decision list is frequently resorting to the default, it will perform on par with the most frequent sense baseline.

The naive Bayes performs on par with the baseline for different reasons. Since a large majority of the unigrams have been seen before, naive Bayes does have probabilities for most of the features in the test for taskA. The problem the naive Bayes system is likely facing is that the training data is almost two thirds objective. For senses other than objective the denominator of $p(f_i |  s)$, using the maximum likelihood estimate, is going to be much larger than the numerator. This is because the other senses do not appear as often so even if $f_i$ is a great indicator of being "neutral" it will not be reflected in the count since "neutral" almost never appears.

Despite this shortcoming the naive Bayes algorithm still manages to outperform the most frequent sense baseline by a small margin. As seen in Table 3 using bigrams and unigrams results in an accuracy $1.4\%$ greater than the baseline. While this is not a very large margin (it means only one more correct tag) it is possibly significant as this is an average across the five runs from fivefold cross-validation. This is comforting that the system can indeed use labels other than the most frequent sense despite the large skew of the corpus. Due to the skew the guesses of the naive Bayes system are almost entirely the most frequent sense. There are only a few departures where the system tags tweet subsets as either positive or negative. These few departures though are accurate, which accounts for the performance above the baseline. This leaves us with the question of why naive Bayes is consistently outperforming the decision list.

Decision lists have an ordered list of rules and simply go through them and execute the first rule that applies. Since rules are based on features which almost always include unigrams it is likely that at least one of the words in the tweet will trigger a rule. The rules in the decision list for taskA are almost entirely objective rules, especially for the highest scoring rules. This is due to the uneven distribution of classes. What this means is that since the algorithm descends down the list of rules it is likely to see an applicable rule at the very top, which will likely be objective. The decision list does not take into account if the next five rules in the decision list were all positive and they would have all been triggered. This is seen in the data by the fact that not a single time in the five runs of the fivefold cross-validation did the decision list classify any tweet subset as being anything other than objective.

Decision lists work well when rules exists with high seperating power, meaning if a rule is triggered the object almost certainly belongs in the specified class. In Twitter though most words have multiple meanings, not all of which belong to the same polarity class. More importantly though is the negation factor. Since any word can be negated, if not directly then in a rhetorical question, no word in the english language will belong solely to one class. i.e. thwarted expectations \footnote{see Pang paper pg 7 for further explanation}. The features which are most indicative of a class, such as extremely positive words like "love" and "awesome", are precisely those features which are most likely to be negated to express the opposite sentiment. The decision list is then expressing $100\%$ confidence when it encounters these rules despite them having a not insignificant margin of error.

On the other hand, naive Bayes allows for much more flexibility. Even if a given word has a strong correlation with being positive the tweet can still be labeled a different sense if a sufficient number of other words are associated with a different polarity. These words do not have to be in close proximity to each other which allows the naive Bayes to take into account context in a natural way. In sharp contrast, the decision list only accounts for context if the included features have some notion of context included in them, which unigrams certainly do not.

Both naive Bayes and the decision list seriously outperformed the subjectivity lexicons. This makes perfect sense as the subjectivity lexicons were using a sentiment dictionary derived from newspaper texts. The writing in newspapers differs markedly from tweets. The most notable difference is the absence of slang, emoticons and abbreviations. The naive Bayes and decision list classifiers will notice the possible skew of some slang words and emoticons towards specific sentiments. This is because the probabilities are being generated from real tweets and so reflect the way the language is actually used. This is exciting as it to some degree saves us the step of having to figure out the sentiment of individual words using some kind of bootstrapping algorithm.

We attempted to rectify the lack of context in decision lists by incorporating bigrams and trigrams. For bigrams or trigrams alone the training data was too small so the decision list resorted to labeling with the most frequent sense every time. However, what was surprising was that combining unigrams and bigrams did not improve the decision list performance, and even made it slightly worse. It seems then that bigrams do not capture context well. This meshes with our realization that often negating words are far removed from the target of their negation. By the fact that the performance did change slightly with the addition of bigrams we know that some bigrams ended up in the decision list and were triggered. One would think that bigrams would be more accurate than unigrams. This failure is surprising until we remember that in the bigram's defense they likely all have very low counts. Since our training set was not very large it is unlikely that any bigram was seen a large number of times. So any given bigram's distribution among the senses could easily be a fluke. Given each bigram's small number of occurences it is probable that a few of the bigrams fell all in one sense. These bigrams would then be added as incredibly accurate rules despite their suspect nature.

Since naive Bayes does not have the same problem with unjustified early commitment the suspect nature of bigrams is not a large problem. Combining bigrams with unigrams in naive Bayes improved performance over using unigrams alone in both tasks. This reflects the fact that in aggregate the bigrams are more accurate than the unigrams. Since the naive Bayes is multiplying so many terms it is not unduly affected by a few incorrect terms. We note though, that when using both bigrams and unigrams the naive Bayes system consistently guesses more words as objective, the most frequent sense. Since we are already performing above the most frequent sense baseline this means that the bigrams are correcting some over-eager unigram tagging. 

Recall the unique smoothing algorithm we used. When we used add-k smoothing our performances significantly dropped across the board. The reason is that our decision list defaults to most frequent sense once the scores fall below a cutoff of $1$. As $V$ gets larger the scores of the rules become lower. This means the decision list resorts to the default more readily, decreasing accuracy. 

An additional quirk of our system which we found is that the naive Bayes model performs much better when using our modified formula. Originally we were using the real naive Bayes formula and multiplying by $P(s)$. We think the performance might have improved when we stopped multiplying by $P(s)$ since this allowed senses other than the most frequent sense to have a chance at being chosen. However, on examination it actually appears that more diverse senses are chosen when $P(s)$ is included. Our system then becomes way more selective about deviating from the most frequent sense when excluding $P(s)$ and subsequently outperforms the baseline.

We are clearly relying on knowledge of the most frequent sense from our training data. However, we believe this is not a problem as the most frequent sense is so much more common than the rest as to suggest that it is a feature of Twitter and not our corpus. Since our systems are fine-tuned specifically for Twitter this is no more of a problem than taking advantage of our knowledge that the tweets we are analyzing are in english.


\section{Conclusion}
In both taskA and taskB the naive Bayes and the decision list are very conservative systems. They guess the most frequent sense unless they are fairly confident that a different sense is correct. This is demonstrated by over $80\%$ of the guesses being in the most frequent sense. This strategy of guessing is fairly succesful as the systems perform at or above the most frequent sense baselines. This is especially noticeable in taskB where both systems significantly outperform the most frequent sense baseline. 

Our subjectivity lexicon performs quite poorly. This is due to the lexicon being very inapplicable due to its being extracted from newspapers. Despite the limited size of our training data both naive Bayes and the decision list were able to learn appropriate rules. The results of training those two systems on progressively more data showed that their performance will scale if we collect more data to train on. 

\subsection{Future Work}
One shortcoming of our sentiment lexicon is that the sentiment words weren't derived from tweets. This is a large problem since word usage changes dramatically from written text to internet micro-blogging. In order to account for slang, emoticons, and other idiosyncracies of Twitter it would be interesting to implement a bootstrapping algorithm~\cite{Riloffbootstrapping}. This would allow us to extract sentiment words directly from training data increasing the accuracy of the sentiment lexicon.

Another difficulty we faced is the uneven distribution of the class. A potential solution to this is to implement the modified naive bayes discussed by \cite{FranknaiveBayes}. This is a large issue as even in taskB we still have one class accounting for over half of all occurrences. The authors show a promising solution involving a normalization step during the smoothing process which reduces the impact of the class skew.

Finally, addressing the issue of negation would significantly improve our performance. Implementing some kind of rule based system could possibly correct for our system thinking "not great" is positive. Many of our errors were due to negations such as the one previously mentioned so even a naive fix for this could lead to improvements.

\bibliographystyle{cs65f12}
\bibliography{cs65f12}
\end{document}

Future idea then maybe tweak decision list so that once it hits a rule it still descends for a while more and if a bunch more rules are triggered it has some way of weighing them in an appropriate manner.


Another interesting thought is how much does knowing the polarity of the tweet as a whole inform our guess of the twet subsets? Are they corellated to any degree, can we use that?
